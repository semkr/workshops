\documentclass[runningheads,a4paper]{llncs}
%
\usepackage{natbib} % bibliography stuff
%
\usepackage{graphicx} % allows for working with images
\DeclareGraphicsExtensions{.pdf,.png,.jpeg,.jpg,.gif} % configures latex to look for the following image extensions
%
\usepackage{setspace} % allows for configuring the linespacing in the document
%\singlespacing
\onehalfspacing
%\doublespacing
%
\usepackage{listings}
%
\usepackage{amsmath}
%
\usepackage{appendix}
%
\usepackage{multirow}
%
\usepackage{booktabs,array,dcolumn}
%
\usepackage{float}
%
\usepackage{eurosym}
%
\usepackage{caption}
\captionsetup[table]{skip=10pt}
\captionsetup{compatibility=false}
\usepackage{subcaption}
%
\usepackage[toc]{glossaries}
\makeglossaries
%
\usepackage{amssymb}
\setcounter{tocdepth}{4}
%
\usepackage{url}
\urldef{\mailsa}\path|info@southeastmakerspace.org|
\newcommand{\keywords}[1]{\par\addvspace\baselineskip
\noindent\keywordname\enspace\ignorespaces#1}

%Optional Package to add PDF bookmarks and hypertext links
\usepackage[pdftex,hypertexnames=false,linktocpage=true]{hyperref}
\hypersetup{colorlinks=true,linkcolor=black,anchorcolor=black,citecolor=black,filecolor=black,urlcolor=black,bookmarksnumbered=true,pdfview=FitB}

\begin{document}
\mainmatter  % start of an individual contribution

% first the title is needed
\title{An Introduction to Arduino \\a low cost digital prototyping platform}

% a short form should be given in case it is too long for the running head
\titlerunning{Introduction to Arduino}
%

% the name(s) of the author(s) follow(s) next
%
% Chinese authors should write their first names(s) in front of
% their surnames. This ensures that the names appear correctly in
% the running heads and the author index.
%
\author{Clive Barry%
%\thanks{Please note that the LNCS Editorial assumes that all authors have used
%the western naming convention, with given names preceding surnames. This determines
%the structure of the names in the running heads and the author index.}%
\and Mogue Carpenter \and Aileen Drohan \and David Kirwan \and Martin Walshe}
%
\authorrunning{Introduction to Arduino}
% (feature abused for this document to repeat the title also on left hand pages)

% the affiliations are given next; don't give your e-mail address
% unless you accept that it will be published
\institute{South East Makerspace,\\Old Printworks, Thomas Hill,\\Waterford City, X91 TW63\\
\mailsa\\
\url{https://www.southeastmakerspace.org}}
%

%
% a more complex sample for affiliations and the mapping to the
% corresponding authors can be found in the file "llncs.dem"
% (search for the string "\mainmatter" where a contribution starts).
% "llncs.dem" accompanies the document class "llncs.cls".
%
%\toctitle{Thesis Proposal}
\tocauthor{SEMS}
\maketitle

%
\begin{abstract}
\begin{figure}
	\centering
	\includegraphics[width=8cm]{images/sems}
\end{figure}
This workshop offers an introduction to the Arduino prototyping platform, useful for artists, hobbiests and the wider maker community. Each participant receives a pack containing an Arduino Uno and components to perform each of 5 simple experiments. No prior coding experience is needed as all code will be provided. The only requirement is the Arduino IDE, and a packed lunch!
\keywords{arduino, digital electronics, prototyping, introduction}
\end{abstract}
%

%%%%%%%%%%%%%%%%%%%%%%%%%%%%%%%%%%%%%%%%%%%%%%%%%%%%%%%%%%%%%%%%%%%%%%%%%%%%%%%%%%%%%%%%%%%%%%%%%%%
%% Acknowledgements
%\newpage
%\chapter*{Acknowledgements}
I'd like to thank my legs, for always supporting me; my arms, who are always by my side and lastly my fingers, I can always count on them.

%

%
\tableofcontents
%

%
%\newpage
%\listoffigures
%\addcontentsline{toc}{chapter}{List of Figures}
%

%
%\newpage
%\listoftables
%\addcontentsline{toc}{chapter}{List of Tables}

%%%%%%%%%%%%%%%%%%%%%%%%%%%%%%%%%%%%%%%%%%%%%%%%%%%%%%%%%%%%%%%%%%%%%%%%%%%%%%%%%%%%%
%

%
\newpage
\chapter*{Introduction}
\addcontentsline{toc}{chapter}{Introduction}

"Over the years \gls{Arduino} has been the brain of thousands of projects, from everyday objects to complex scientific instruments. A worldwide community of makers - students, hobbyists, artists, programmers, and professionals - has gathered around this open-source platform, their contributions have added up to an incredible amount of accessible knowledge that can be of great help to novices and experts alike."~\citep{arduino-15} 

%
\begin{figure}[ht]
	\centering
	\includegraphics[width=8cm]{images/01}
	\caption{Arduino Uno R3 \citep{wikipedia-13}}
	\label{fig:arduino_uno_r3}
\end{figure}
%

\section*{Where might an Arduino be used?}
\addcontentsline{toc}{section}{Where might an Arduino be used?}
A few contrived examples of where one might use an Arduino in order to automate some process:

\subsection*{Automatic Dog's Water Bowl}
An dog owner wants to ensure her pet is never left without water. She attaches a system for measuring the water level in the dog's bowl. Her \gls{Arduino} is programmed to measure this value every 5 minutes. If this level falls below a certain value, a valve is opened and a water pump is activated to fill up the water bowl. It then sends an SMS to the owner to let her know that the dog is in safe hands.

\begin{itemize}
	\item Read inputs from a water level sensor
	\item Control a valve which lets water flow
	\item Control speed of a water pump 
	\item Send SMS to owner about giving the dog water
\end{itemize}


\subsection*{Bird Table Camera}
A rising social media ornithologist wishes to share pictures from all the visitors to the bird table in his garden. He mounts an infra-red movement sensor on the bird table attached to an \gls{Arduino} which is configured to record an image and send it to Twitter. His neighbours marvel at how many crows he's feeding.

\begin{itemize}
	\item Read inputs from a movement sensor
	\item Control a camera shutter
	\item Transmit image back to PC
	\item Send tweet with picture of the bird table visitor
\end{itemize}


\subsection*{Fingerprint Door Lock}
A student is sick of forgetting his keys and being locked out of his house. He uses a fingerprint scanner and an Arduino to make a biometric fingerprint door lock. He needs only scan his thumb print now and the door will unlock.

\begin{itemize}
	\item Read inputs from a fingerprint sensor
	\item Compares the finger print against an authorised fingerprint
	\item Records the time and date a finger was pressed on the scanner
	\item Makes audio error tone if the fingerprint was invalid
	\item If valid fingerprint it unlocks the door
\end{itemize}


\newpage
\section*{Workshop Aims}
\addcontentsline{toc}{section}{Workshop Aims}
In this workshop the aim is to give you a crash course in digital electronics, and providing you the basic skills to start using the Arduino micro-controllers in your future projects.

\begin{description}
	\item[Workshop Requirements] \hfill \\
	Each person will require the following:
	\begin{itemize}
		\item PC, either Linux, Mac or Windows can be used
		\item Arduino IDE pre-installed (Internet at the makerspace is flaky!)
		\item A sambo to keep you going
	\end{itemize}
	
	\item[Learning Outcomes] \hfill \\
	Each person will leave with:
	\begin{itemize}
		\item Arduino starter kit
		\item Crash course in digital electronics
		\item Confidence to use Arduino in future projects
	\end{itemize}
\end{description}

\subsection*{Arduino Starter Kit Contents}
The Arduino starter kit contains the following components, which we will be making use of during the workshop.

\begin{itemize}
	\item 1 $\times$ Arduino Compatible R3 Uno
	\item 1 $\times$ Breadboard
	\item 16 $\times$ jumper wires various colours
	\item 20 $\times$ 5mm LED's assorted colours
	\item 10 $\times$ 10k ohm resistors
	\item 10 $\times$ 330ohm resistors
	\item 1 $\times$ RGB LED
	\item 1 $\times$ photo resistor
	\item 2 $\times$ push buttons
	\item 1 $\times$ temperature sensor
	
\end{itemize}


\newpage
\section*{Basic Circuit Theory}
\addcontentsline{toc}{section}{Basic Circuit Theory}

In an electrical circuit there is a fundamental relationship between voltage, current and resistance and it is explained by Ohm’s Law~\citep{et-15}.

\subsection*{Voltage}
Voltage, (SI Unit: $V$ - Volts) is the potential energy of an electrical supply stored in the form of an electrical charge. Voltage can be thought of as the force that pushes electrons through a conductor and the greater the voltage the greater is its ability to push the electrons through a given circuit~\citep{et-15}.

%
\begin{figure}[ht]
	\centering
	\includegraphics[width=7cm]{images/02}
	\caption{Voltage Symbols \citep{et-15}}
	\label{fig:voltage_symbols}
\end{figure}
%

\subsection*{Current}
Current, (SI Unit: $A$ - Ampere) is the movement or flow of electrical charge and is measured in Amperes. It is the continuous and uniform flow (called a drift) of electrons (the negative particles of an atom) around a circuit that are being pushed by the voltage source~\citep{et-15}.

%
\begin{figure}[ht]
	\centering
	\includegraphics[width=4cm]{images/03}
	\caption{Current Symbols \citep{et-15}}
	\label{fig:current_symbols}
\end{figure}
%

\newpage
\subsection*{Resistance}
Resistance, (SI Unit: $\Omega$ - Ohms) of a circuit is its ability to resist or prevent the flow of current (electron flow) through itself making it necessary to apply a greater voltage to the electrical circuit to cause the current to flow again. Note that Resistance cannot be negative in value only positive~\citep{et-15}.

%
\begin{figure}[ht]
	\centering
	\includegraphics[width=8cm]{images/04}
	\caption{Resistance Symbols \citep{et-15}}
	\label{fig:resistance_symbols}
\end{figure}
%

\subsection*{Ohm's Law}
The following equation Ohm's Law explains the relationship between Voltage, Current and Resistance for an electrical circuit.

%
\begin{figure}[ht]
	\centering
	\begin{equation}
	V = I \times R
	\end{equation}
	\begin{equation}
	I = \frac{V}{R}
	\end{equation}
	\begin{equation}
	R = \frac{V}{I} 
	\end{equation}
	\caption{Ohm's Law}
	\label{fig:ohms_law_equation}
\end{figure}
%


\begin{itemize}
	\item Voltage, current \& resistance
	\item Ohms law
	\item Resistor, capacitor, LED, photo-resistor
	\item Breadboard
	\item Digital vs analogue signals
\end{itemize}


\section*{Basic Arduino Coding Concepts}
\addcontentsline{toc}{section}{Basic Arduino Coding Concepts}

\begin{itemize}
	\item Variables
	\item setup() function
	\item loop() function
\end{itemize}


\section*{Phenakistoscope Creation}
\addcontentsline{toc}{section}{Phenakistoscope Creation}


\citep{kalif-15-a} \citep{kalif-15-b}


%
\begin{table}
	\centering
	\begin{tabular}{p{4cm} l}
		\toprule
		Column 1 & Column 2\\ \midrule
		x & 1 \\
		y & 2 \\
		z & 3 \\
		\bottomrule
	\end{tabular}
	\caption{Table Caption}
	\label{tab:table_label}
\end{table}
%
\chapter*{Experiment 1 - Blink}
\addcontentsline{toc}{chapter}{Experiment 1 - Blink}
This example is what one might call the \textit{hello world} example for \gls{Arduino} sketches. We wire up the experiment as shown in the diagram fig:~\ref{fig:exp1_blink}. And upload the \gls{Sketch} code in the next section on page:~\pageref{sketch:exp1}.

%
\begin{figure}[ht]
	\centering
	\includegraphics[width=12cm]{images/07}
	\caption{LED Blink \citep{fritzing-15}}
	\label{fig:exp1_blink}
\end{figure}
%

When the Arduino boots, the led should flash on for a second, then off for a second and repeat.

\newpage
\section*{Sketch Code}
\label{sketch:exp1}
\begin{lstlisting}
/*
Blink
Turns on an LED on for one second, then off for one second, repeatedly.

This example code is in the public domain.
*/

// the setup function runs once when you press reset or power the board
void setup() {
  // initialize digital pin 13 as an output.
  pinMode(13, OUTPUT);
}

// the loop function runs over and over again forever
void loop() {
  // turn the LED on 
  // (HIGH is the voltage level)
  digitalWrite(13, HIGH);
	
  //wait for 1000 milliseconds
  delay(1000);
	
  // turn the LED off by making 
  // the voltage LOW
  digitalWrite(13, LOW);    
	            
  // wait for 1000 milliseconds              
  delay(1000);
}
\end{lstlisting}
\chapter*{Experiment 2}
\addcontentsline{toc}{chapter}{Experiment 2}

Outline each experiment in a separate chapter
\chapter*{Experiment 3}
\addcontentsline{toc}{chapter}{Experiment 3}

Outline each experiment in a separate chapter
\chapter*{Experiment 4 - MP3 Player}
\addcontentsline{toc}{chapter}{Experiment 4 - MP3 Player}
We wire up the experiment as shown in the diagram fig:~\ref{fig:exp4_mp3}. And upload the sketch code in the next section on page:~\pageref{sketch:exp4}.

%
\begin{figure}[ht]
	\centering
	\includegraphics[width=12cm]{images/11}
	\caption{Read a buttons input \citep{dfrobot-15a}}
	\label{fig:exp4_mp3}
\end{figure}
%

This example requires that a library be loaded in order to use, a copy of which is available in the companion code associated with this workshop, see also:~\citep{dfrobot-15b}. Instructions for installing a library can be found at the following url: https://www.arduino.cc/en/Guide/Libraries\#.UxU8mdzF9H0 

\newpage
\section*{Sketch Code}
\label{sketch:exp4}
\begin{lstlisting}
/*
Sample code to work with the DFPlayer Mini MP3 Player

Author: David Kirwan
Licence: Apache 2.0
*/
// libraries
#include <SoftwareSerial.h>
#include <DFPlayer_Mini_Mp3.h>

boolean playing; // Create a variable called playing

void setup () {
  Serial.begin (9600);
  mp3_set_serial (Serial);
  delay(1);
  mp3_set_volume (7);
  playing = false;
}

void loop () {
  if(!playing){
    playing = true;
    mp3_play ();
  }
}

/*
  mp3_play (); //start play
  mp3_play (5); //play "mp3/0005.mp3"
  mp3_next (); //play next
  mp3_prev (); //play previous
  mp3_set_volume (uint16_t volume); //0~30
  mp3_set_EQ (); //0~5
  mp3_pause ();
  mp3_stop ();
  void mp3_get_state ();
  void mp3_get_volume ();
  void mp3_get_u_sum ();
  void mp3_get_tf_sum ();
  void mp3_get_flash_sum ();
  void mp3_get_tf_current ();
  void mp3_get_u_current ();
  void mp3_get_flash_current ();
  void mp3_single_loop (boolean state);
  void mp3_DAC (boolean state);
  void mp3_random_play ();
*/

\end{lstlisting}
\chapter*{Experiment 5 - clap4Led}
\addcontentsline{toc}{chapter}{Experiment 5 - clap4Led}
We wire up the experiment as shown in the diagram fig:~\ref{fig:exp5_microphone}. And upload the sketch code in the next section on page:~\pageref{sketch:exp5}.

%
\begin{figure}[ht]
	\centering
	\includegraphics[width=8cm]{images/21}
	\caption{Microphone Activated Circuit}
	\label{fig:exp5_microphone}
\end{figure}
%

The microphone module is polarised, which means it has a positive and negative terminal which must be correctly placed in the circuit in order to function as expected. Please examine the diagram fig:~\ref{fig:exp5_microphone_polarity} on page:~\pageref{fig:exp5_microphone_polarity} to correctly identify which leg is positive and which is negative.

When this circuit is complete, the LED onboard the Arduino will flash when the microphone detects sound.

%
\begin{figure}[ht]
	\centering
	\includegraphics[width=12cm]{images/22}
	\caption{Microphone Module Polarity}
	\label{fig:exp5_microphone_polarity}
\end{figure}
%


\newpage
\section*{Sketch Code}
\label{sketch:exp5}
\begin{lstlisting}
/*
Sample code to work with a Thermistor

Author: Anton Krug
Licence: Apache 2.0
*/

//global variables
int readValue;
int maximumValue;
int minimumValue; 


// will reset maximum and minimum values only when we call it
void resetMaximumAndMinimum() 
{
  maximumValue = 0;             // set maximum to lowest value possible
  minimumValue = 32767;         // set minimum to highest value possible
}


// this runs only once on the startup
void setup() 
{
  // initialize built-in LED light at digital pin 13 as an output
  pinMode(13, OUTPUT);           
  resetMaximumAndMinimum();
}


// the loop runs over and over again forever
void loop() 
{
  // read current value from the microphone
  readValue = analogRead(A0);     

  // if read value is smaller than minimum then set it as the new minimum
  if (readValue < minimumValue)   
  {
    minimumValue = readValue;     
  }

  // if read value is bigger than maximum then set it as the new maximum
  if (readValue > maximumValue)   
  {
    maximumValue = readValue;     
  }

  // Change the 10 constant to adjust the sensitivity.
  //   To 20 if you want the light triggered on louder claps   
  //   Ti 5  if you want the light triggered on quieter noises 
  //   Feel free to experiment with the values.
  if ( (maximumValue - minimumValue) > 10 ) 
  {
    digitalWrite(13, HIGH);       // turn the LED on (HIGH is the voltage level)
    delay(2 * 1000);              // wait for 2 seconds (2 * 1000ms)
    digitalWrite(13, LOW );       // turn the LED off by making the voltage LOW

    // if we wouldn't clear the max & min then after first trigger it would get 
    // triggered every single time no matter what read values were
    resetMaximumAndMinimum();     
  }
}

\end{lstlisting}
\newpage
\chapter*{Thanks for taking part!}
\addcontentsline{toc}{chapter}{Thanks for taking part!}

Do you want to know more? Of course you do! \citet{fried-12} has developed a number of tutorials suitable for the beginner wishing to learn more about the \gls{Arduino} system. These tutorials can be accessed at the following URI: http://www.ladyada.net/learn/arduino/

I hope you enjoyed this workshop and your time spent here at the South East Makerspace! We would be delighted to welcome new faces! For information on SEMS see:

\begin{description}
	\item[Facebook] http://www.facebook.com/SouthEastMakerSpace
	\item[FAQ] https://wiki.southeastmakerspace.org/faq
	\item[Google+] http://plus.google.com/u/0/108025738894009906004/
	\item[How to join] https://wiki.southeastmakerspace.org/how\_to\_join\_sems
	\item[Mailing List] http://lists.southeastmakerspace.org/mailman/listinfo/sems-general
	\item[Secure Contact] https://www.southeastmakerspace.org/contact
	\item[Twitter] http://twitter.com/SEMakerSpace
\end{description} 





%

%
%%%%%%%%%%%%%%%%%%%%%%%%%%%%%%%%%%%%%%%%%%%%%%%%%%%%%%%%%%%%%%%%%%%%%%%%%%%%%%%%%%%%%
%
%
\newglossaryentry{Arduino}
{
  name={Arduino},
  description={an open-source electronics platform based on easy-to-use hardware and software. It's intended for anyone making interactive projects. https://www.arduino.cc},
  sort=Arduino
}
%

%
\newglossaryentry{LED}
{
	name={LED},
	description={a light emitting diode is a p-n junction diode which emits light when activated. https://en.wikipedia.org/wiki/Light-emitting\_diode},
	sort=LED
}
%

%
\newglossaryentry{Sketch}
{
	name={Sketch},
	description={a sketch is the name that Arduino uses for a program. Once compiled the resulting firmware can be installed on the Arduino device. https://www.arduino.cc/en/Tutorial/Sketch},
	sort=Sketch
}
%
\renewcommand*{\glsclearpage}{}
\printglossaries
%

%
%\appendix
%\chapter*{Appendix}
\addcontentsline{toc}{chapter}{Appendices}

%

%
\bibliographystyle{plainnat}
\bibliography{bibliography/bibtex}
\addcontentsline{toc}{chapter}{Bibliography}
%

\end{document}


